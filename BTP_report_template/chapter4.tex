\chapter{Tree Decomposition}

\textbf{Definition} : A tree decomposition of a graph G is a pair (T,B) where T is a tree 
and $B: V(T) \to 2^{V(G)}$
satisfies the following 
\begin{itemize}
    \item For each vertex v in G, there is a node x in V(T) such that v is in B(x) \vspace{-0.4cm}
    \item For each edge e=\{u, v\} in G, there is a node x in V(T) such that u are v are in B(x) \vspace{-0.4cm}
    \item For each vertex v in G, the set \{x $\in$ V(T) : v $\in$ B(x) \} induces a connected graph
\end{itemize}

\section{Simple Tree Decomposition} 
\textbf{Definition} : A simple tree decomposition (T,B) is a tree decomposition in which
there is no pair of distinct nodes x 
and y in T such that B(x) $\subseteq$ B(y)

\subsection{Finding a simple tree decomposition}
\fbox{
 \parbox{15cm}{
Question : Given a graph G and a tree decomposition pair (T,B) where T is a tree 
and B $B: V(T) \to 2^{V(G)}$, find a simple tree decomposition (T',B') of G
}
}\vspace{0.6cm}


First, we have to verify whether the given tree decomposition is a tree decomposition or not.

\begin{enumerate}
    
\item To verify the first rule of tree decomposition, check for every vertex in graph if there is 
 a bag that contains the vertex. If its not true for every vertex, then it is not a tree decomposition.   

\item To verify the second rule of tree decomposition, check for each edge in graph if there is 
 a bag that contains both the end points of the edge. If its not true for every edge,
 then it is not a tree decomposition.

\item To verify the third rule of tree decomposition, for every vertex v, we do the following checking.
First we mark all the bags that contains v. Then we will start a modified breadth first search
from any arbitrary node, which will only explore the children which contains v and mark these 
vertices as visited. Otherwise we ignore the vertex. 
Once the search is finished, if the visited vertices are same as the vertices which we marked
initially then we say that the rule 3 is followed.  
\end{enumerate}

If it is a tree decomposition, we now further verify if its a simple tree decomposition or not.
\begin{enumerate}
\item To verify the rule of simple tree decomposition, check 
if there are any bags such that one is the subset of the another. 
\end{enumerate}

If there are no such bags, then we say that the given tree decomposition is a simple tree decomposition 
Otherwise we remove a node from the tree which has the bag that is subset of other bag and 
add the edges between the children of removed vertex and the superset node. 

\vspace{0.3cm}
Total running time of the algorithm is $O(n^{\alpha})$

\section{Optimum Tree Decomposition}
\textbf{Width of a tree decomposition T} = w(T) = max \{ $|B(x)|$ : x $\in$ V(T)\} - 1 \\
\textbf{Treewidth of G}, tw(G) = min \{w(T) : T is a tree decomposition of G\} 


\textbf{Definition} : An optimal tree decomposition of a graph G is a tree 
decomposition of G of width tw(G) 

\subsection{Finding a optimum tree decomposition}
\fbox{
 \parbox{15cm}{
    Question : Given a graph G, compute an optimum tree decomposition of G
 }
}\vspace{0.6cm}

To find the optimum tree decomposition, we enumerate on the all the possible tree decomposition
pairs and find the minimum width.

\hspace*{1.5cm} We start the size of the tree T from 1, and increase till V(G)-1.
For a particular size n of the tree, a tree consists of n-1 edges. We need to pick n-1 edges
which forms a tree from the n*(n-1)/2 edges of the complete graph consisting of n vertices. So total 
atmost 
$\binom{(n*(n-1)/2)}{n-1}$ number of tree combinations are possible. 

\hspace*{1.5cm} For a fixed tree, we have at most $2^{(n^2)}$ choices of B. 

\hspace*{1.5cm} After choosing a pair of (T,B), we verify whether it is a tree decomposition. If yes, then we will 
update minimum treewidth if the new width value is minimum.

Total running time of the algorithm is $2^{O(n^2)}$


\section{Dominating set}

\textbf{Definition} : 
A dominating set in Graph G is a set S of vertices such that N(S)=V(G)
or every vertex not in S is adjacent to at least one member of S.

\subsection{Dominating set on Trees}
\fbox{
 \parbox{15cm}{
    Instance : A tree T and an integer K 

    Question : Does there exist a dominating set of T of size at most K?
 }
}\vspace{0.6cm}

To solve this problem we use bottom up approach of dynamic programming.
We solve the subparts of the problem and use them to get the final solution.

\vspace{0.5cm}
Root T at an arbitrary vertex. Let that root vertex be \textbf{root} \\
For a vertex v, let $T_v$ denote the subtree of T rooted at v 

\begin{itemize}    
\item Let $\Gamma(v)$ denote the minimum possible size of a dominating set in $Tv$ \vspace{-0.4cm}
\item Let $\Lambda(v)$ denote the minimum possible size of a dominating set in $T_v$ 
that dominates every vertex in $T_v-v$ \vspace{-0.4cm}
\item Let $\Delta(v)$ denote the minimum possible size of a dominating set in $T_v$ that 
contains v
\end{itemize}

$\Gamma(root)$ is the final solution which we need. To get this value, we start
solving for the $\Gamma(v)$, $\Lambda(v)$, $\Delta(v)$ values from the leaf nodes, which
 will be used further as we 
go till root of the tree. 

If v is a leaf, then 

\hspace*{2cm} $\Gamma(v)$ = 1

\hspace*{2cm} $\Lambda(v)$ = 0

\hspace*{2cm}$\Delta(v)$ = 1


If v has $v_1, v_2, ....., v_q$ as its children, then 

\hspace*{2cm} $\Delta(v)$ = 1 + $\Lambda(v_1)+\Lambda(v_2)+....+\Lambda(v_q)$

$\Delta(v)$ definitely contain v in dominating set, so we need not cover children of v. 
But we should cover all the grand children of v.

\hspace*{2cm} $\Lambda(v)$ = min\{ $\Gamma(v_1) + \Gamma(v_2) + .... + \Gamma(v_q)$,  
$1 + \Lambda(v_1)+\Lambda(v_2)+....+\Lambda(v_q)$ \}

$\Lambda(v)$ may or may not contain v in dominating set. If we include v in solution,
then we need to cover all the grand children of v. If we dont include v in solution, then
we should cover all the children and grand children of v.

\hspace*{2cm} $\Gamma(v)$ = min\{ $1 + \Lambda(v_1)+\Lambda(v_2)+....+\Lambda(v_q)$,
min\{$\Delta(v_i) + \Sigma_{i \neq j} \Gamma(v_j) : i \in [q] $ \} \}

Total running time of the algorithm is O(n).





% \chapter{Algorithm II}

\section{Iterative Compression of Feedback Vertex Set}

Iterative compression is an algorithmic technique for the design of fixed-parameter tractable algorithms,
in which one element (such as a vertex of a graph) is added to the problem in each step, 
and a small solution for the problem prior to the addition is used to help find a small 
solution to the problem after the step.

\begin{enumerate}
        \item Start with a subgraph induced by a vertex set S of size k, and a solution X 
        that equals S itself.
        \item While S $\neq$ V, perform the following steps:
        \begin{itemize}
                \item Let v be any vertex of V \ S, and add v to S
                \item Test whether the (k + 1)-vertex solution Y = X ∪ {x} to S can
                 be compressed to a k-vertex solution.
                \item If it cannot be compressed, abort the algorithm: the input graph has no 
                k-vertex solution.
                \item Otherwise, set X to the new compressed solution and continue
                 the loop.
        \end{itemize}
\end{enumerate}



Instance : Given a graph G and K+1 size solution S \\
Question: Does there exist a feedback vertex set of G of size at most k?

Let the solution set of size atmost k be R.

Divide the set s into 2 parts. One part which contains the vertices from R and other part X which does 
contain vertices from R.

Total $2^{k+1}$ choices of S $\cap$ R

Let r number of vertices be in X.

Now our aim is to find a set of at most r-1 vertices from X and Y (i.e. G-S).

Reduction :
Delete all the vertices of degree at most 1 in G

Reduction :
If there exists any vertex v in Y such that G(X $\cup$ v) contains a cycle,
then include this vertex in the solution, remove this vertex and decrease the parameter by 1.

Reduction :
If there is a vertex v in Y of degree 2 in G such
that at least one neighbor of v in G is from Y, then delete this vertex
and make its neighbors adjacent

Since Y is a forest, Y has a vertex v of degree at most 1. 

v has $\geq2$ neighbours in Y, otherwise above reduction rules would have been applied.
And also those two neighbours will be in different components of Y, otherwise above reduction
rule would have been applied. 

Here we have two possibilities. Either v is in solution or v is not in solution.
Branch into these possibilities. \\
If v is in solution, then the reduced instance is (X,Y-$\{v\}$,r-2) \\
If v is not in solution, then the reduced instance is (X $ \cup \{v\}$ ,Y-$\{v\}$,r-1)

In one branch, r decreases by 1 and in another branch connected components of x decrease by 1.

Let I = r + \#C(X), where \#C(X) is the number of connected components of X. \\
I $\leq$ k+1+k $\leq$ 2k+1

Let T(I) denote the number of leaves in the tree rooted at instance with measure I.

T(I) $\leq$ 2T(I-1) if I $\geq$ 1 
\vspace{-0.5cm}
\begin{verbatim}  
        1 otherwise
\end{verbatim}
\vspace{-0.5cm}

Total running time for disjoint compression algorithm is $O(2^{2k})$

If there exists an algorithm solving Disjoint Feedback Vertex Set in
 Tournaments in time g(k)·n , then there exists an algorithm
solving Feedback Vertex Set in Tournaments Compression in time
$\Sigma_{i=0}{k}   \binom{k+1}{i} g(k-i) n^{O(1)}$

If g(k) = $4^k$, then the running time of the algorithm is $O(5^k)$  


\section{Randomized Feedback Vertex Set}
Instance : Given a graph G and a integer k \\
Question: Does there exist a feedback vertex set of G of size at most k? \\
Parameter: k

Reduction Rule 1: Delete isolated vertices \\
Reduction Rule 2: Delete degree-1 vertices \\
Reduction Rule 3: If there is a loop at a vertex v, delete v from the graph
and reduce the parameter by 1 \\
Reduction Rule 4: If there is an edge with multiplicity $>$ 2, reduce it to 2 \\
Reduction Rule 5: Short circuit degree-2 vertices 

After applying the reduction rules, graph will have minimum degree 3.

If G is graph with minimum degree $\geq$ 3, then number of edges
incident to any feedback vertex set S is $> |E(G)|/2$

Let feedback vertex set of G be S and H = G-S \\
Let E(X) be edges incident to S, E(G) be edges in graph G, E(H) be edges in subgraph H, 
E(H,S) be edges between H and S.

We need to prove that $|E(X)|$ $\geq$ $|E(G)|/2$ \\
$|E(X)| >  |E(G)|/2$ is equivalent to $|E(X)| > |V(H)|$ \\
$2|E(X)| > |E(G)| $ \\
$2|E(X)| > |E(X)|+|E(H)| $ \\
$|E(X)| > |E(H)| $ \\
$|E(X)| > |V(H)|$ \\

Let $V_{\leq 1}$, $V_2$ and $V_{\geq 3}$ denote the set of vertices in V(H) such that they
have degree at most 1, exactly 2, and at least 3 in H respectively. \\
$|E(X)| \geq |E(H,S)|$ \\
$|E(X)| \geq 2|V_{\leq 1}| + |V_2|$ \\
$|E(X)| > |V_{\leq 1}| + |V_2| + |V_{\geq 3}|$ \\
$|E(X)| > |V(H)|$ 

Feedback Vertex Set can be solved in randomized polynomial
time, with success probability at least $4^{-k}$

Run the following algorithm $4^k$ times. If none of the
executions return yes, then declare that (G,k) is a no-instance. Other wise, declare that
(G,k) is an yes-instance.

Initialize the solution set to empty. Then apply the above mentioned reduction rules to the graph.
Now graph will have minimum degree 3. Pick a edge uniformly at random from the edges in the reduced
instance. Now, pick a endpoint from the edge picked previously uniformly at random. Add this
endpoint to the solution set and delete that from the reduced instance. And continue the process till
our parameter is positive. Now check if the solution set obtained from the algorithm is a feedback
vertex set or not. If yes, we return it as solution.

Total running time of the algorithm is $O(4^k)$

Proof of correctness : \\
If (G,k) is a no-instance then algorithm always outputs no.
Suppose (G,k) is an yes-instance and F is a $\leq$ k feedback vertex set.
Let H=G-F \\
Pr (an edge in E(F) $\cup$ E(H,F) is chosen) $>$ 1/2 \\
Pr (a vertex from F is chosen) $>$ 1/2 * 1/2 = 1/4 \\
Pr(S=F) $>$ (1/4) k \\
Pr(Algorithm says no) $< (1-(1/4)^k)^{(4)^k} \leq 1/e $ \\
Pr(Algorithm says yes) $> 1- 1/e \geq 1/2 $





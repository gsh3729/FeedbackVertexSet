\chapter{Tree Decomposition}

% \section{Tree Decomposition}
A tree decomposition of a graph G is a pair (T,B) where T is a tree and $B: V(T) \to 2^{V(G)}$
satisfies the following \\
For each vertex v in G, there is a node x in V(T) such that v is in B(x) \\
For each edge e=\{u, v\} in G, there is a node x in V(T) such that u are v are in B(x) \\
For each vertex v in G, the set \{x $\in$ V(T) : v $\in$ B(x) \} induces a connected graph \\

\section{Checking Tree Decomposition}
Question : Given a graph G and a pair (T,B) where T is a tree and B $B: V(T) \to 2^{V(G)}$, 
 check if (T,B) is a tree decomposition of G

First, we will check whether for every vertex in graph is there a bag that contains
the vertex.

Thereafter, we will check for each edge whether both the end points of the edge
are in the same bag.

To verify the third rule, for every vertex v, we do the following checking.
First we mark all the bags that contains v. Then we will start a modified breadth first search
from any arbitrary node, which will only explore the children which contains v and mark these 
vertices as visited. Otherwise we ignore the vertex. 
Once the search is finished, if the visited vertices are same as the vertices which we marked
initially then we say that the rule 3 is followed.  

\section{Simple Tree Decomposition}
A simple tree decomposition (T,B) is one where there is no pair of distinct nodes x 
and y in T such that B(x) $\subseteq$ B(y)

Question : Given a graph G and a tree decomposition (T,B) of G, 
check if (T,B) is a simple tree decomposition.

First we will verify whether (T,B) is a tree decomposition or not. If yes,
then we will try to see if there are any bags such that one is the subset of the another.

\section{Convert to Simple Tree Decomposition}
Question : Given a graph G and a tree decomposition (T,B) of G, 
find a simple tree decomposition (T',B') of G such that w(T') $\leq$ w(T).

To convert a tree decomposition in to simple tree decomposition, we need to remove 
all the nodes of the tree which are subsets of other nodes and connected to the superset node
directly by an edge.

Find out the bag x which is subset of bag y. Then check if there is edge between x and y in the
tree. If yes, then remove the node containing bag x from the tree and add the connections between 
the y and children of x.  

\section{Optimum Tree Decomposition}
Width of a tree decomposition T = w(T) = max \{ $|B(x)|$ : x $\in$ V(T)\} - 1 \\
Treewidth of G, tw(G) = min \{w(T) : T is a tree decomposition of G\} 


Definition : An optimal tree decomposition of G is a tree decomposition of G of width tw(G) \\
Question : Given a graph G, compute an optimum tree decomposition of G

To find the optimum tree decomposition, we enumerate on the all the possible tree decomposition
pairs and find the minimum width.

We start the size of the tree T from 1, and increase till V(G)-1.
For a particular size n of the tree, a tree consists of n-1 edges. We need to pick n-1 edges
which forms a tree from the n*(n-1)/2 edges of the complete graph consisting of n vertices. So total 
atmost 
$\binom{(n*(n-1)/2)}{n-1}$ number of tree combinations are possible. 

For a fixed tree, we have at most $2^{(n^2)}$ choices of B. 

After choosing a pair of (T,B), we verify whether it is a tree decomposition. If yes, then we will 
update minimum treewidth if the new width value is minimum.


\section{Dominating set}

\subsection{Definition}
A dominating set in Graph G is a set S of vertices such that N(S)=V(G)
or every vertex not in S is adjacent to at least one member of S.

\subsection{Dominating set on Trees}
Problem Statement \\
Instance : A tree T and an integer K \\
Question : Does there exist a dominating set of T of size at most K?

To solve this problem we use bottom up approach of dynamic programming.
We solve the subparts of the problem and use them to get the final solution.\\
Root T at an arbitrary vertex. Let that root vertex be \textbf{root} \\
For a vertex v, let $T_v$ denote the subtree of T rooted at v \\ 
Let $\Gamma(v)$ denote the minimum possible size of a dominating set in $Tv$ \\
Let $\Lambda(v)$ denote the minimum possible size of a dominating set in $T_v$ 
that dominates every vertex in $T_v-v$ \\
Let $\Delta(v)$ denote the minimum possible size of a dominating set in $T_v$ that 
contains v

$\Gamma(root)$ is the final solution which we need. To get this value, we will start
solving for the above values from the leaf nodes, which will be used further as we 
go till root of the tree. 

If v is a leaf, then 
$\Gamma(v)$ = 1,
$\Lambda(v)$ = 0,
$\Delta(v)$ = 1.


If v has $v_1, v_2, ....., v_q$ as its children, then 

$\Delta(v)$ = 1 + $\Lambda(v_1)+\Lambda(v_2)+....+\Lambda(v_q)$\\
$\Delta(v)$ will definitely contain v in its dominating set, so we need not cover children of v. 
But we should cover all the grand children of v.

$\Lambda(v)$ = min\{ $\Gamma(v_1) + \Gamma(v_2) + .... + \Gamma(v_q)$,  
$1 + \Lambda(v_1)+\Lambda(v_2)+....+\Lambda(v_q)$ \}

$\Gamma(v)$ = min\{ $1 + \Lambda(v_1)+\Lambda(v_2)+....+\Lambda(v_q)$,
min\{$\Delta(v_i) + \Sigma_{i \neq j} \Gamma(v_j) : i \in [q] $ \} \}



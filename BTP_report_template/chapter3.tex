\chapter{Feedback Vertex Set}

\section{Branching Feedback Vertex Set}
Definition : Feedback Vertex Set of a graph is a set of vertices that has at
least one vertex of every cycle

Instance: An undirected graph G and an integer k \\
Question: Does there exist a feedback vertex set of G of size at most k?
Parameter: k


Reduction Rule 1 : Delete isolated vertex v \\
Resulting instance : (G-v, k) \\
(G,k) is an yes-instance iff (G-v,k) is an yes-instance

(G,k) is an yes-instance if (G-v,k) is an yes-instance\\
Adding a isolated vertex v to graph G-v does not change the size of feedback vertex set as \\
it does not from any cycle.

(G-v,k) is an yes-instance if (G,k) is an yes-instance\\
Removing a isolated vertex v to graph G does not change the size of vertex cover as 
it won't be in any cycle of graph.

Reduction Rule 2 : Delete degree-1 vertex \\
Resulting instance : (G-v, k) \\
(G,k) is an yes-instance iff (G-v,k) is an yes-instance

(G-v,k) is an yes-instance if (G,k) is an yes-instance \\
Removing a degree-1 vertex does not change the size of feedback vertex set as it won't be part 
of any cycle

(G,k) is an yes-instance if (G-v,k) is an yes-instance\\
Adding a degree-1 vertex does not change the size of feedback vertex set as it won't be part 
of any cycle

Reduction Rule 3: If there is a loop at a vertex v, delete v from the graph
and reduce the parameter by 1 

Reduction Rule 4: If there is an edge with multiplicity > 2, reduce it to 2

Reduction Rule 5: Short circuit degree-2 vertices \\
Reduction instance : (G',k) \\
(G,k) is an yes-instance iff (G',k) is an yes-instance

Let v be the degree-2 vertex and x,y be its neighbours. \\
(G,k) is an yes-instance iff (G',k) is an yes-instance \\
Introducing an vertex v between x and y (v-x edge and v-y edge), does not change the size
of the feedback vertex set.

(G',k) is an yes-instance iff (G,k) is an yes-instance \\
Any cycle through v also goes through x and y. \\
So removing vertex v does not change the size of the feedback vertex set. \\

Now after applying these reduction rules, graph contains minimum degree 3. 

Every FVS of size $\leq$ k contains at least one vertex from the top 3k vertices after sorting
the vertices according to their degree. 

Let $v_1, v_2, ......, v_n$ be the descending order of vertices according to their degree. \\
$v_h$ = \{$v_1,v_2,....,v_{3k}$ \} 

Let us assume that there is a feedback vertex set X of size at most k such that $X \cap v_{h} = \emptyset$.

Graph F = G-X is a forest  \\
Number of edges in F is atmost $|V(G)|-|X|-1$. \\
Every edge of E(G)-E(F) is incident to a vertex of X. \\
$\Sigma_{v \in X} d(v) + |V(G)|-|X|-1 \geq |E(G)|$ \\
$\Sigma_{v \in X} (d(v)-1) \geq |E(G)|-|V(G)|+1$ 

For every v $\in$ X, d(v) is at most the minimum of vertex degrees from $v_h$ \\
$\Sigma_{i=1}^{3k}(d(v_i)-1) \geq 3*(\Sigma_{v \in X}(d(v_)-1) \geq 3*(|E(G)|-|V(G)|+1)$ \\
$\Sigma_{i>3k}(d(v_i)-1) \geq \Sigma_{v \in x}$     \\
$\Sigma_{i=1}^{n}(d(v_i)-1) \geq 4*(|E(G)|-|V(G)|+1)$ \\
$\Sigma_{i=1}^{n}(d(v_i)) = 2|E(G)|$ \\
$2|E(G)| - |V(G)| \geq 4*(|E(G)|-|V(G)|+1)$ \\
$2|E(G)| < 3|V(G)|$ \\
This contradicts the fact that every vertex of G is of minimum degree 3.

From these 3k vertices, at least one will be in solution. So we have 3k branches each branch
choosing one vertex from the $v_h$. \\
Reduced instance : (G-v, k-1)

If the reduced instance graph is empty and still k$\geq$0, then we return as yes-instance.
If the k$<$0, then we return as no-instance.

Total running time of the algorithm is $O((3k)^k)$


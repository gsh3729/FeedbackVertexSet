\chapter{Feedback Vertex Set}

\textbf{Definition:} Feedback Vertex Set of a graph is a set of vertices that has at
least one vertex of every cycle


\section{Branching Algorithm}

\fbox{
 \parbox{15cm}{Instance : An undirected graph G and an integer k

 Question : Does there exist a feedback vertex set of G of size at most k?

Parameter : k}
}\vspace{0.6cm}

\textbf{Reduction Rule 1:} Delete isolated vertex v 

Resulting instance : (G-v, k) 

\begin{theorem}    
(G,k) is an yes-instance iff (G-v,k) is an yes-instance
\end{theorem}

\textbf{Proof:}

\hfill \vspace{-1.5cm}
      \begin{center}
              (G,k) is an yes-instance if (G-v,k) is an yes-instance 
       \end{center} \vspace{-0.2cm}
Adding a isolated vertex v to graph G-v does not change the size of feedback vertex set as 
it does not from any cycle.

\newpage
\begin{center}
    (G-v,k) is an yes-instance if (G,k) is an yes-instance    
\end{center} \vspace{-0.2cm}
Removing a isolated vertex v to graph G does not change the size of vertex cover as 
it won't be in any cycle of graph.

\textbf{Reduction Rule 2:} Delete degree-1 vertex 

Resulting instance : (G-v, k) 

\begin{theorem}
    (G,k) is an yes-instance iff (G-v,k) is an yes-instance
\end{theorem}

\textbf{Proof:}

\hfill \vspace{-1.5cm}
      \begin{center}
                (G-v,k) is an yes-instance if (G,k) is an yes-instance 
       \end{center} \vspace{-0.2cm}
Removing a degree-1 vertex does not change the size of feedback vertex set as it won't be part 
of any cycle

\begin{center}
    (G,k) is an yes-instance if (G-v,k) is an yes-instance    
\end{center}\vspace{-0.2cm}
Adding a degree-1 vertex does not change the size of feedback vertex set as it won't be part 
of any cycle

\textbf{Reduction Rule 3:} If there is a loop at a vertex v, delete v from the graph
and reduce the parameter by 1 

\textbf{Reduction Rule 4:} If there is an edge with multiplicity $>$ 2, reduce it to 2

\textbf{Reduction Rule 5:} Short circuit degree-2 vertices 

Reduction instance : (G',k) 

\begin{theorem}
(G,k) is an yes-instance iff (G',k) is an yes-instance    
\end{theorem}

\textbf{Proof:}

\hfill \vspace{-1.5cm}
      \begin{center}
(G,k) is an yes-instance iff (G',k) is an yes-instance 
       \end{center} \vspace{-0.2cm}

Let v be the degree-2 vertex and x,y be its neighbours.

Introducing an vertex v between x and y (v-x edge and v-y edge), does not change the size
of the feedback vertex set.

\begin{center}
    (G',k) is an yes-instance iff (G,k) is an yes-instance     
\end{center}\vspace{-0.2cm}

Any cycle through v also goes through x and y. 
So removing vertex v does not change the size of the feedback vertex set. 


Now after applying these reduction rules, graph contains minimum degree-3. 

\begin{theorem}
Every FVS of size $\leq$ k contains at least one vertex from the top 3k vertices after sorting
the vertices according to their degree. 
\end{theorem}

\textbf{Proof}:

Let $v_1, v_2, ......, v_n$ be the descending order of vertices according to their degree. 

\hspace*{2cm} $v_h$ = \{$v_1,v_2,....,v_{3k}$ \} 

Let us assume that there is a feedback vertex set X of size at most k 
such that $X \cap v_{h} = \emptyset$.

Graph F = G-X is a forest  

Number of edges in F is atmost $|V(G)|-|X|-1$. 

Every edge of E(G)-E(F) is incident to a vertex of X. 

\hspace*{2cm} $\Sigma_{v \in X} d(v) + |V(G)|-|X|-1 \geq |E(G)|$ 

\hspace*{2cm} $\Sigma_{v \in X} (d(v)-1) \geq |E(G)|-|V(G)|+1$ 

For every v $\in$ X, d(v) is at most the minimum of vertex degrees from $v_h$ 

\hspace*{2cm} $\Sigma_{i=1}^{3k}(d(v_i)-1) \geq 3*(\Sigma_{v \in X}(d(v_)-1) \geq 3*(|E(G)|-|V(G)|+1)$ 

\hspace*{2cm} $\Sigma_{i>3k}(d(v_i)-1) \geq \Sigma_{v \in x}$     

\hspace*{2cm} $\Sigma_{i=1}^{n}(d(v_i)-1) \geq 4*(|E(G)|-|V(G)|+1)$ 

\hspace*{2cm} $\Sigma_{i=1}^{n}(d(v_i)) = 2|E(G)|$ 

\hspace*{2cm} $2|E(G)| - |V(G)| \geq 4*(|E(G)|-|V(G)|+1)$ 

\hspace*{2cm} $2|E(G)| < 3|V(G)|$ 

This contradicts the fact that every vertex of G is of minimum degree 3.

From these 3k vertices, at least one will be in solution. So we have 3k branches each branch
choosing one vertex from the $v_h$. 

Reduced instance: (G-v, k-1)

If the reduced instance graph is empty and still k$\geq$0, then we return as yes-instance.
If the k$<$0, then we return as no-instance.

Total running time of the algorithm is $O^*((3k)^k)$




\section{Iterative Compression Algorithm}

\hspace*{1.5cm} Iterative compression is an algorithmic technique for the design of
 fixed-parameter tractable algorithms,
in which one element (such as a vertex of a graph) is added to the problem in each step, 
and a small solution for the problem prior to the addition is used to help find a small 
solution to the problem after the step.

\begin{enumerate}
        \item Start with a subgraph induced by a vertex set S of size k, and a solution X 
        that equals S itself.
        \item While S $\neq$ V, perform the following steps:
        \begin{itemize}
                \item Let v be any vertex of V \ S, and add v to S \vspace{-0.4cm}
                \item Test whether the (k + 1)-vertex solution Y = X ∪ {x} to S can
                 be compressed to a k-vertex solution. \vspace{-0.4cm}
                \item If it cannot be compressed, abort the algorithm: the input graph has no 
                k-vertex solution. \vspace{-0.4cm}
                \item Otherwise, set X to the new compressed solution and continue
                 the loop.
        \end{itemize}
\end{enumerate}


\fbox{
 \parbox{15cm}{Instance: Given a graph G and K+1 size solution S

 Question: Does there exist a feedback vertex set of G of size at most k?

Parameter : k}
}\vspace{0.6cm}

Let the solution set of size atmost k be R.

Divide the set s into 2 parts. One part which contains the vertices from R and other part X which does 
contain vertices from R.

Total there are $2^{k+1}$ choices of S $\cap$ R

Let r number of vertices be in X.

As k+1-r are already in solution, now our aim is to find a set of at most r-1 vertices 
from Y (i.e. G-S).

 \textbf{Reduction rule 1:}
Delete all the vertices of degree at most 1 in G

 \textbf{Reduction rule 2:}
If there exists any vertex v in Y such that G(X $\cup$ v) contains a cycle,
then include this vertex in the solution, remove this vertex and decrease the parameter by 1.

 \textbf{Reduction rule 3:}
If there is a vertex v in Y of degree 2 in G such
that at least one neighbor of v in G is from Y, then delete this vertex
and make its neighbors adjacent.

\vspace{0.3cm}
Since Y is a forest, Y has a vertex v of degree at most 1. 

\vspace{0.3cm}
v has $\geq2$ neighbours in Y, otherwise above reduction rules would have been applied.
And also those two neighbours will be in different components of Y, otherwise above reduction
rule would have been applied. 

\vspace{0.3cm}
Here we have two possibilities. Either v is in solution or v is not in solution.
Branch into these possibilities. 

If v is in solution, then the reduced instance is (X,Y-$\{v\}$,r-2) 

If v is not in solution, then the reduced instance is (X $ \cup \{v\}$ ,Y-$\{v\}$,r-1)

\vspace{0.3cm}
In one branch, r decreases by 1 and in another branch connected components of x decrease by 1.

\vspace{0.3cm}
Let I = r + \#C(X), where \#C(X) is the number of connected components of X. 

\hspace*{2cm}I $\leq$ k+1+k $\leq$ 2k+1

Let T(I) denote the number of leaves in the tree rooted at instance with measure I.

\[
    T(I)=
    \left\{
    \begin{array}{lr}
       2T(I-1) & if I \geq 1 \\
      1& otherwise 
    \end{array}
    \right\} 
\]

Total running time for disjoint compression algorithm is $O^*(2^{2k})$

If there exists an algorithm solving Disjoint Feedback Vertex Set in
 Tournaments in time g(k)·n , then there exists an algorithm
solving Feedback Vertex Set in Tournaments Compression in time
$\Sigma_{i=0}^{k}   \binom{k+1}{i} g(k-i) n^{O(1)}$

If g(k) = $4^k$, then the running time of the algorithm is $O^*(5^k)$  


\section{Randomized Algorithm}
\fbox{
 \parbox{15cm}{Instance: Given a graph G and a integer k 

 Question: Does there exist a feedback vertex set of G of size at most k?

Parameter : k}
}\vspace{0.6cm}


\textbf{Reduction Rule 1:} Delete isolated vertices 

\textbf{Reduction Rule 2:} Delete degree-1 vertices 

\textbf{Reduction Rule 3:} If there is a loop at a vertex v, delete v from the graph
and reduce the parameter by 1 

\textbf{Reduction Rule 4:} If there is an edge with multiplicity $>$ 2, reduce it to 2 

\textbf{Reduction Rule 5:} Short circuit degree-2 vertices 

After applying the reduction rules, graph contains minimum degree 3.

\begin{theorem}
        If G is graph with minimum degree $\geq$ 3, then number of edges
        incident to any feedback vertex set S is $> |E(G)|/2$        
\end{theorem}

\textbf{Proof:}

Let feedback vertex set of G be S and H = G-S 

Let E(X) be edges incident to S, E(G) be edges in graph G, E(H) be edges in subgraph H, 
E(H,S) be edges between H and S.

\hspace*{2cm} We need to prove that $|$E(X)$| > |$E(G $|/2$

\hspace*{1.6cm} $|E(X)| >  |E(G)|/2$ is equivalent to $|E(X)| > |V(H)|$ 

\hspace*{2.6cm} $2|E(X)| > |E(G)| $ 

\hspace*{2.6cm} $2|E(X)| > |E(X)|+|E(H)| $ 

\hspace*{2.6cm} $|E(X)| > |E(H)| $ 

\hspace*{2.6cm} $|E(X)| > |V(H)|$ 

Let $V_{\leq 1}$, $V_2$ and $V_{\geq 3}$ denote the set of vertices in V(H) such that they
have degree at most 1, exactly 2, and at least 3 in H respectively. 

\hspace*{2cm} $|E(X)| \geq |E(H,S)|$ 

\hspace*{2cm} $|E(X)| \geq 2|V_{\leq 1}| + |V_2|$ 

\hspace*{2cm} $|E(X)| > |V_{\leq 1}| + |V_2| + |V_{\geq 3}|$ 

\hspace*{2cm} $|E(X)| > |V(H)|$ 

Feedback Vertex Set can be solved in randomized polynomial
time, with success probability at least $4^{-k}$

\vspace{0.3cm}
Run the following algorithm $4^k$ times. If none of the
executions return yes, then declare that (G,k) is a no-instance. Other wise, declare that
(G,k) is an yes-instance.

\hspace*{1.5cm} Initialize the solution set to empty. Then apply the above mentioned reduction rules to the graph.
Now graph will have minimum degree 3. Pick a edge uniformly at random from the edges in the reduced
instance. Now, pick a endpoint from the edge picked previously uniformly at random. Add this
endpoint to the solution set and delete that from the reduced instance. And continue the process till
our parameter is positive. Now check if the solution set obtained from the algorithm is a feedback
vertex set or not. If yes, we return it as solution.

\vspace{0.3cm}
Total running time of the algorithm is $O^*(4^{k})$

\newpage
\textbf{Proof:}

If (G,k) is a no-instance then algorithm always outputs no.

Suppose (G,k) is an yes-instance and F is a $\leq$ k feedback vertex set.

Let H=G-F 

\hspace*{1.5cm} Pr (an edge in E(F) $\cup$ E(H,F) is chosen) $>$ 1/2 

\hspace*{1.5cm} Pr (a vertex from F is chosen) $>$ 1/2 * 1/2 = 1/4 

\hspace*{1.5cm} Pr(S=F) $>$ (1/4)k 

\hspace*{1.5cm} Pr(Algorithm says no) $< (1-(1/4)^k)^{(4)^k} \leq 1/e $ 

\hspace*{1.5cm} Pr(Algorithm says yes) $> 1- 1/e \geq 1/2 $



\pagenumbering{arabic}\hspace{3mm}

\chapter{Vertex Cover}

% \section{Vertex Cover}
\textbf{Definition} : Vertex cover of a graph is a set of vertices 
such that each edge of the graph is incident to at least one vertex of the set.

\vspace{-0.8cm}
\section{Branching Algorithm}
\fbox{
 \parbox{15cm}{Instance : A graph G on n vertices, m edges and integer k. 

 Question : Does G have a vertex cover of size atmost k? 

Parameter : k}
}

\vspace{0.6cm}

\textbf{Reduction Rule 1} : Delete isolated vertex v \\
Resulting instance : (G-v, k) 

\begin{theorem}
       (G,k) is an yes-instance iff (G-v,k) is an yes-instance  
\end{theorem}

% \begin{proof} 
\textbf{Proof:}

\hfill \vspace{-1.5cm}
      \begin{center}
              (G,k) is an yes-instance if (G-v,k) is an yes-instance 
       \end{center} \vspace{-0.2cm}
Adding a isolated vertex v to graph G-v does not change the size of vertex cover as 
it does not have any edge with other vertices.
\newpage
       \begin{center}
              (G-v,k) is an yes-instance if (G,k) is an yes-instance               
       \end{center}
Removing a isolated vertex v to graph G does not change the size of vertex cover as 
it does not have any edge with other vertices.
% \end{proof}

\textbf{Reduction Rule 2} : Delete high degree vertices($\geq$k) \\
Resulting instance : (G-v, k-1) 

\begin{theorem}
       (G,k) is an yes-instance iff (G-v,k-1) is an yes-instance        
\end{theorem}

% \begin{proof}
\textbf{Proof:}

\hfill \vspace{-1.5cm}
       \begin{center}
(G-v,k-1) is an yes-instance if (G,k) is an yes-instance               
       \end{center}
If G contains a vertex v of degree more than k, then v should be in
every vertex cover of size at most k. So we can remove vertex v and decrease the parameter by 1.

       \begin{center}
              (G,k) is an yes-instance if (G-v,k-1) is an yes-instance               
       \end{center} 
If we add a vertex of degree $\geq$ k to graph, then either this vertex should be in solution 
or all the neighbours of v which are greater than k should be in solution. The later case is not
possible. So this vertex should definitely in vertex cover. So adding a vertex of degree $\geq$ k leads
to increase of parameter by 1. 
% \end{proof}

\textbf{Reduction Rule 3}: Add neighbour of degree-1 vertices to solution \\
Degree-1 vertex will only have an edge with one other vertex. Either it or its neighbour needs to be in
solution. Adding the neighbour to solution will be more useful as it may cover some more edges in the
graph.

\textbf{Reduction Rule 4} : For degree-2 vertices \\
\textbf{Case 1} : 
\begin{theorem}
       If neighbours of degree-2 vertex has an edge between them, then the resulting instance 
       is (G', k-2) 
\end{theorem}
% \begin{proof}
\textbf{Proof:}

       Let the degree-2 vertex be v and its neighbour be x and y. \\
       Any vertex cover has at least 2 from {v,x,y}. \\
       If v is included, then v covers 2 edges and to cover edge between x and y we require one of them. \\
       If v is not included, then both of its neighbours should be included. \\
       Resulting instance : (G-\{v,x,y\}, k-2)   
% \end{proof}


\textbf{Case 2} : 
\begin{theorem}
       If neighbours of degree-2 vertex does not have an edge between them, then the
resulting instance is (G', k-1)
\end{theorem}

% (G,k) is an yes-instance iff (G',k-1) is an yes-instance
% \begin{proof}
\textbf{Proof:}

\hfill \vspace{-1.5cm}
       \begin{center}
              (G',k-1) is an yes-instance if (G,k) is an yes-instance               
       \end{center}
Let the degree-2 vertex be v and its neighbour be x and y. Let z be a vertex having edges with 
neighbours of x and y apart from v. 

Either v is in minimum vertex cover or x and y are in minimum vertex cover. \\
\hspace*{2cm} If v is in minimum vertex cover, then 
 both x and y will not be in minimum vertex cover. As we are not including x and y,
all the neighbours of x and y will be in minimum vertex cover. So we can delete this vertex
from the graph and reduce the parameter by 1.

\hspace*{2cm} If x and y are in minimum vertex cover, then it will cover the edges of v. So now we need to cover only neighbours
of x and y apart from v. To cover these edges we can remove x and y from the graph and add vertex 
z
       \begin{center}
              (G,k) is an yes-instance if (G',k-1) is an yes-instance               
       \end{center}
Either z in T or z is not in T. 

If z is not in minimum vertex cover, then neighbours of x and y will be in solution and edges of vertex 
needs to be covered. So include v to solution, this will increase the parameter by only 1. 

If z is in minimum vertex cover, then we can delete this vertex and include x and y to the solution.
As x and y will cover all the neighbours of z and also v.
% \end{proof}

Now, let the final reduced instance be (G,k) and G has a minimum degree-3 vertex. 

Find a vertex v $\in$ V(G) of maximum degree in G. 

Either v or N(v) in the vertex cover. 

2 branches of (G,k) are (G-v, k-1) and (G-N(v), k-$|N(G)|$). 

If the reduced instance graph is empty and still k≥0, then we return as yes-instance. If the
k$<$0, then we return as no-instance.

Apply preprocessing reduction rules to the new reduced instance and continue the same process. 

Running time = (number of nodes in the search tree) * (time taken at each node) 

Time taken at each node is in order of n. 

Number of nodes in the search tree = 2l-1, where l is the number of leaves. 

So Running time is bounded by the number of leaves in the tree. 

Let T(k) denote the no. of leaves in the tree rooted at instance with parameter k 

\[
    T(K)=
    \left\{
    \begin{array}{lr}
       T(k-1) + T(k-3)& if k \geq 3 \\
      1& otherwise 
    \end{array}
    \right\} 
\]

Total running time of the algorithm is $O(n^3 + 1.465^k*k^3)$
